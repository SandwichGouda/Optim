Complexity theory is an important domain of theoretical computer science, the most famous problem of which is probably the \og P = NP \fg problem. It asks whether or not NP problems are in P. Though we can, for many problems, proove that they \textit{are indeed} in a given problem complexity class, it is always very difficult to proove that a given problem is \textit{not} in some class.

We present the general theory, investigate the definitions along with their philosophical meanings, explore different complexity classes (co-NP, NPC, NPH, PH, ...) and their relationships. Next, we consider various algorithmic problems, study the relationships and reductions between them and some algorithms to solve them.

\subsection{Turing machines and complexity}

Turing machines were originally introduced by A. Turing \cite{Turing1936} \cite{Turing1992}

\subsubsection{The polynomial hierarchy}
optim ; 
Once $P$ and $NP$ have been defined,