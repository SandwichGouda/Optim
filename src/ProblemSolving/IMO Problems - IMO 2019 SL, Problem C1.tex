\begin{problem}[IMO 2019 SL - Problem C1]
    The infinite sequence $a_0, a_1, a_2, \ldots$ of (not necessarily different) integers has the following properties: 
    \[
    0 \leq a_i \leq i \quad \text{for all integers } i \geq 0,
    \]
    and
    \[
    \binom{k}{a_0} + \binom{k}{a_1} + \cdots + \binom{k}{a_k} = 2^k
    \quad \text{for all integers } k \geq 0.
    \]
    Prove that all integers $N \geq 0$ occur in the sequence (that is, for all $N \geq 0$, there exists $i \geq 0$ with $a_i = N$).
\end{problem}

\section*{Solution}

We prove by induction on $k$ that every initial segment of the sequence, $a_0, a_1, \ldots, a_k$, consists of the following elements (counted with multiplicity, and not necessarily in order), for some $\ell \geq 0$ with $2\ell \leq k+1$:
\[
0, 1, \ldots, \ell - 1, 0, 1, \ldots, k - \ell.
\]

For $k = 0$ we have $a_0 = 0$, which is of this form. Now suppose that for $k = m$ the elements $a_0, a_1, \ldots, a_m$ are 
\[
0, 0, 1, 1, 2, 2, \ldots, \ell - 1, \ell - 1, \ell, \ell + 1, \ldots, m - \ell - 1, m - \ell
\]
for some $\ell$ with $0 \leq 2\ell \leq m + 1$. It is given that
\[
\binom{m+1}{a_0} + \binom{m+1}{a_1} + \cdots + \binom{m+1}{a_m} + \binom{m+1}{a_{m+1}} = 2^{m+1},
\]
which becomes

\begin{multline*}
\left( \binom{m+1}{0} + \binom{m+1}{1} + \cdots + \binom{m+1}{\ell - 1} \right) \\
+ \left( \binom{m+1}{0} + \binom{m+1}{1} + \cdots + \binom{m+1}{m - \ell} \right)
+ \binom{m+1}{a_{m+1}} = 2^{m+1}.
\end{multline*}

Or, using $\binom{m+1}{i} = \binom{m+1}{m+1 - i}$, that

\begin{multline*}
\left( \binom{m+1}{0} + \binom{m+1}{1} + \cdots + \binom{m+1}{\ell - 1} \right) \\
+ \left( \binom{m+1}{m+1} + \binom{m+1}{m} + \cdots + \binom{m+1}{\ell+1} \right)
+ \binom{m+1}{a_{m+1}} = 2^{m+1}.
\end{multline*}

On the other hand, it is well known that
\[
\sum_{i=0}^{m+1} \binom{m+1}{i} = 2^{m+1},
\]
and so, by subtracting, we get
\[
\binom{m+1}{a_{m+1}} = \binom{m+1}{\ell}.
\]

From this, using the fact that the binomial coefficients $\binom{m+1}{i}$ are increasing for $i \leq \frac{m+1}{2}$ and decreasing for $i \geq \frac{m+1}{2}$, we conclude that either $a_{m+1} = \ell$ or $a_{m+1} = m + 1 - \ell$. In either case, $a_0, a_1, \ldots, a_{m+1}$ is again of the claimed form, which concludes the induction.

As a result of this description, any integer $N \geq 0$ appears as a term of the sequence $a_i$ for some $0 \leq i \leq 2N$.