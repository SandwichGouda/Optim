\subsection{The simplex}
Usual introductions to the simplex algorithm start the following way : consider a company that has $n$ products to sell $p_1$, ..., $p_n$. It shall produce nonnegative (not necessarily integral) amounts $x_1,...,x_n$ of each product. To do so, the company makes use of $m$ machines, each can respectively run $b_1, ..., b_m$ minutes per month. Each product must pass though each machine, during an amount proportional to the quantity that must be produced : for each $1\le i\le n$ and each $1\le j\le m$, producing an amount $x_j$ of product $p_j$ requires machine $i$ to run $a_{ij}x_j$ minutes. So, the amounts $x_j$ must satisfy the contraints
\[
    \forall i=1,...,m, \sum_{j=1}^n a_{ij} x_j \leqslant b_i.
\]
Recall that the produced amounts can only be nonnegative, so we must also have
\[
    \forall j=1,...,n, x_j \geqslant 0.
\]
Finally, product $p_j$ will be sold at cost $c_j$. The company then wants to maximize its profit
\[
    \sum_{j=1}^n c_j x_j
\]
The motivates the Simplex problem.

\begin{problem}
Simplex problem

Inputs : real numbers $x_1,\ldots,x_n, c_1,\ldots,c_n$

Output : $ \displaystyle \max \sum_i x_i c_i$
\end{problem}

\subsection{The simplex, matrix version}
\subsection{Duality in linear optimization}