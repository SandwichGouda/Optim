\documentclass[12pt,openany,oneside]{book}

\usepackage[utf8]{inputenc}
\usepackage[T1]{fontenc}
\usepackage{lmodern}
\usepackage[english]{babel}
\newcommand{\og}{«\:}
\newcommand{\fg}{»}

\usepackage{amsmath}
\usepackage{amssymb}
\usepackage{amsfonts}
\usepackage{amsthm}

\usepackage{hyperref}
\usepackage{cite}
\usepackage{fancyhdr}
\usepackage{graphicx}
\usepackage{svg}
\usepackage{geometry}
\usepackage{parskip}

\usepackage{indentfirst}
\setlength{\parindent}{15pt}
\setlength{\parskip}{6pt}

\usepackage{algorithm}
\usepackage{algpseudocode} % For algorithmic environment

\newgeometry{top=1.5in,bottom=1.5in,left=1.2in,right=1.2in}
\pagestyle{plain}

\renewcommand{\chaptermark}[1]{\markboth{\chaptername\ \thechapter.\ #1}{}} % This actually sets \leftmark
\renewcommand{\sectionmark}[1]{\markright{\thesection.\ #1}} % This sets \rightmark
\fancyhead[LE,RO]{\nouppercase{\rightmark}}
\fancyhead[LO,RE]{\nouppercase{\leftmark}}

\title{\textbf{Optimization and beyond}}
\author{William DRIOT}
\date{2025-2026}

\begin{document}
\maketitle

\tableofcontents

\newpage

\chapter*{Introduction}
\pagestyle{fancy}

Throughout the years, I have learnt to learn. 

One of the strongest conclusion that I came up to is that the best way (for me) to learn, is to write down things that I have understood, clearly, with my words, in the way that I have done here in this document. This enables good understanding, memorizing, and to ensure thorough study of the state of the art, whatever topic I am deep diving into.

Besides ensuring a great understanding of notions, this document also provides proof to anyone as to how determined, passionate and committed I can be.

During this gap year, I have, although not as much as theoretically possible, more time than ever to perform such deep and thorough diving. Last year, O. Fercoq's course on optimization made me go nuts, and I have likely decided that this branch of mathematics will be the one I ought to dedicate my life to.

I am excited to start my jour building strong foundations as I set out on this long adventuremy life journey with the 

\textit{Vivent les mathématiques}, and let's get to it.

% I have studied cont. opt., comb. opt. ; 
% \input{content/ContinuousOptimization.tex}
% \input{content/CombinatorialOptimization.tex}
% \input{content/Algorithmics.tex}
% \input{content/ProblemSolving.tex}

\part{Continuous optimization}

\chapter{Linear optimization}

\section{The simplex}
\subsection{The simplex}
Usual introductions to the simplex algorithm start the following way : consider a company that has $n$ products to sell $p_1$, ..., $p_n$. It shall produce nonnegative (not necessarily integral) amounts $x_1,...,x_n$ of each product. To do so, the company makes use of $m$ machines, each can respectively run $b_1, ..., b_m$ minutes per month. Each product must pass though each machine, during an amount proportional to the quantity that must be produced : for each $1\le i\le n$ and each $1\le j\le m$, producing an amount $x_j$ of product $p_j$ requires machine $i$ to run $a_{ij}x_j$ minutes. So, the amounts $x_j$ must satisfy the contraints
\[
    \forall i=1,...,m, \sum_{j=1}^n a_{ij} x_j \leqslant b_i.
\]
Recall that the produced amounts can only be nonnegative, so we must also have
\[
    \forall j=1,...,n, x_j \geqslant 0.
\]
Finally, product $p_j$ will be sold at cost $c_j$. The company then wants to maximize its profit
\[
    \sum_{j=1}^n c_j x_j
\]
The motivates the Simplex problem.
\fbox{
\begin{minipage}{\dimexpr\textwidth-2\fboxsep-2\fboxrule}
Simplex problem

Inputs : real numbers $x_1,\ldots,x_n, c_1,\ldots,c_n$

Output : $\max \sum_i x_i c_i$
\end{minipage}
}



\subsection{The simplex, matrix version}
\subsection{Duality in linear optimization}

\section{Karmarkar's method}

\chapter{Convex optimization}

\chapter{Nonlinear optimization}

\section{Uncontrained nonlinear optimization}
\section{Constrained nonlinear optimization}
\subsection{Lagrange}
\subsection{Karush, Kuhn and Tucker theory}
\subsection{Contrained convex optimization}
\subsubsection{Frank and Wolfe's method}
\subsubsection{The cutting plane method}
\section{Lagrangian relaxation}
\section{Gradient methods}
\subsubsection{Fixed step gradient method}
\subsubsection{Optimal step gradient method}
\subsubsection{Conjugate gradient descent}
\subsection{Méthode de Newton}
\subsection{KKT Theory}
\subsection{Le Lagrangien}
\section{Optimisation continue stochastique}
\subsection{Méthode du gradient stochastique}
\subsection{Mirror descent}

\chapter{Stochastic optimization}

\section{Stochastic gradient descent}
\subsection{Setting and method}
\subsection{Theoretical bounds}
\subsection{Choosing the step}
\section{Particle SWARM optimization}

\chapter{Recent topics in optimization}

\section{Polynomial optimization}
% https://wangjie212.github.io/jiewang/research/lectures.pdf
\section{Quantum noncommutative polynomial optimization}
% ncpol3sdpa, examples, incluing the minecraft mobfarm
\section{Trust region algorithms}
% Yuan, Y. "A review of trust region algorithms for optimization" in ICIAM 99: Proceedings of the Fourth International Congress on Industrial & Applied Mathematics, Edinburgh, 2000 Oxford University Press, USA.
% Yuan, Y. "Recent Advances in Trust Region Algorithms", Math. Program., 2015
\subsection{The smoothed duality gap as a stopping criterion}
\section{Jacobi's algorithm (misc)}

\part{Discrete algorithmics}

\chapter{Complexity}

\part{Combinatorial optimization}

\part{Optimal transport}

\chapter{Optimal transport : general  results}

\chapter{Algorithmic approaches to optimal transport}

\section{Reduction to the simplex problem}
\section{The Eulerian point of view}
\section{Monge-Ampère's equation and applications}

\part{Problem solving}

\chapter{International Mathematics Olympiads (IMO) problems}

\section{IMO 2018 SL - Problem C1}
\section{IMO 2018 SL - Problem C2}
\section{IMO 2018 SL - Problem C3}
\section{IMO 2018 SL - Problem C4}
\section{IMO 2018 SL - Problem C5}
\section{IMO 2018 SL - Problem C6}

\section{IMO 2019 SL - Problem C1}
\section{IMO 2019 SL - Problem C2}
\section{IMO 2019 SL - Problem C3}
\section{IMO 2019 SL - Problem C4}
\section{IMO 2019 SL - Problem C5}
\section{IMO 2019 SL - Problem C6}

\chapter{Competitive programming problems}

\chapter{Graph problems}
% Un sujet ENS
\section{Combinatorics problems}

\end{document}
