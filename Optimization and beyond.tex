\documentclass[12pt,openany,oneside]{book}
%
\usepackage[utf8]{inputenc}
\usepackage[T1]{fontenc}
\usepackage{lmodern}
\usepackage[english]{babel}
\newcommand{\og}{«\:}
\newcommand{\fg}{»\:}
%
\usepackage{amsmath}
\usepackage{amssymb}
\usepackage{amsfonts}
\usepackage{amsthm}
%
\usepackage{algorithm}
\usepackage{algpseudocode} % For algorithmic environment
%
\usepackage{hyperref}
\usepackage{cite}
%
\usepackage{graphicx}
\usepackage{svg}
%
% \usepackage{parskip}
% \usepackage{indentfirst}
%
\usepackage{tcolorbox}
\newtcolorbox{problembox}{
    width=0.9\textwidth,
    colback=white, % Background color
    colframe=black, % Frame color
    % coltitle=black, % Color of the title text
    % colupper=red, % Color of the main text
    % colbacktitle=blue, % Background color of the title bar
    arc=8pt, % Rounded corners
    boxrule=1pt, % Frame thickness
    boxsep=3pt, % space between text and border
    left=6pt, % Left padding
    right=6pt, % Right padding
    top=6pt, % Top padding
    bottom=6pt, % Bottom padding
}
%
\usepackage{geometry}
\newgeometry{top=1.5in,bottom=1.5in,left=1.2in,right=1.2in}
%
\usepackage{fancyhdr} % MUST BE AFTER LOADING GEOMETRY AND GEOMETRY SETTINGS !!
\setlength{\headheight}{15pt}
\pagestyle{plain}
%
\renewcommand{\chaptermark}[1]{\markboth{\chaptername\ \thechapter.\ #1}{}} % This actually sets \leftmark
\renewcommand{\sectionmark}[1]{\markright{\thesection.\ #1}} % This sets \rightmark
\fancyhead[R]{\nouppercase{\rightmark}}
\fancyhead[L]{\nouppercase{\leftmark}}
%
\newenvironment{problem}{\begin{center}\begin{problembox}}{\end{problembox}\end{center}}
\newenvironment{examples}{\textbf{Examples.}\begin{itemize}}{\end{itemize}}
%
\theoremstyle{definition}
\newtheorem{definition}{Definition}
\newtheorem{theorem}{Theorem}
\newtheorem{corollary}{Corollary}
\newtheorem{notation}{Notation}
\newtheorem{proposition}{Proposition}
\newtheorem{remark}{Remark}
\newtheorem{hypothesis}{Hypothesis}
\newtheorem{example}{Example}
%
\numberwithin{definition}{section}
\numberwithin{theorem}{section}
\numberwithin{corollary}{section}
\numberwithin{proposition}{section}
\numberwithin{notation}{section}
\numberwithin{remark}{section}
\numberwithin{hypothesis}{section}
\numberwithin{example}{section}
%
\renewcommand\thesubsubsection{\thesubsection.\alph{subsubsection})}
\setcounter{secnumdepth}{3}
\setcounter{tocdepth}{3}
%
\title{\textbf{Optimization and beyond}}
\author{William DRIOT}
\date{2025-2026}
%
% Conventions for this book :
% 1. Environments should be written this way
%   \begin{...}
%       ...
%   \end{...}
% 2. The terms defined in a definition should be in italics.
% 3. Definition should have names : \begin{definition}[Deterministic Turing machine] ... \end{definition}
% 4. Definition names should use singular names.
% 5. We should comply with the rule that says that equations should be accordingly punctuated : 
%   \[
%     \forall x \in E, ... ,
%   \]
%   or
%   \[
%     \forall x \in E, ... .
%   \]
\begin{document}

\maketitle

\tableofcontents

\setlength{\parindent}{15pt}
\setlength{\parskip}{6pt}

\newpage

\chapter*{Introduction}
\pagestyle{fancy}

Throughout the years, I have learnt to learn. 

One of the strongest conclusion that I came up to is that the best way (for me) to learn, is to write down things that I have understood, clearly, with my words, in the way that I have done here in this document. This enables good understanding, memorizing, and to ensure thorough study of the state of the art, whatever topic I am deep diving into.

Besides ensuring a great understanding of notions, this document also provides proof to anyone as to how determined, passionate and committed I can be.

During this gap year, I have, although not as much as theoretically possible, more time than ever to perform such deep and thorough diving. Last year, R. A. Dragomir and  O. Fercoq's lectures on optimization made me go nuts, and I have likely decided that this branch of mathematics will be the one I ought to dedicate my life to.

I am excited to start my journey by building the strong foundations laid out in this document

\textit{Vivent les mathématiques !}

% I have studied cont. opt., comb. opt. ; 
% \input{src/ContinuousOptimization.tex}
% \input{src/CombinatorialOptimization.tex}
% \input{src/Algorithmics.tex}
% \input{src/ProblemSolving.tex}

\part{Continuous optimization}

\chapter{Linear optimization}

\section{The simplex}
\subsection{The simplex}
Usual introductions to the simplex algorithm start the following way : consider a company that has $n$ products to sell $p_1$, ..., $p_n$. It shall produce nonnegative (not necessarily integral) amounts $x_1,...,x_n$ of each product. To do so, the company makes use of $m$ machines, each can respectively run $b_1, ..., b_m$ minutes per month. Each product must pass though each machine, during an amount proportional to the quantity that must be produced : for each $1\le i\le n$ and each $1\le j\le m$, producing an amount $x_j$ of product $p_j$ requires machine $i$ to run $a_{ij}x_j$ minutes. So, the amounts $x_j$ must satisfy the contraints
\[
    \forall i=1,...,m, \sum_{j=1}^n a_{ij} x_j \leqslant b_i.
\]
Recall that the produced amounts can only be nonnegative, so we must also have
\[
    \forall j=1,...,n, x_j \geqslant 0.
\]
Finally, product $p_j$ will be sold at cost $c_j$. The company then wants to maximize its profit
\[
    \sum_{j=1}^n c_j x_j
\]
The motivates the Simplex problem.

\begin{problem}[pb]
Simplex problem

Inputs : real numbers $x_1,\ldots,x_n, c_1,\ldots,c_n$

Output : $ \displaystyle \max \sum_i x_i c_i$
\end{problem}

\subsection{The simplex, matrix version}
\subsection{Duality in linear optimization}

\section{Karmarkar's method}

\chapter{Convex optimization}

\chapter{Nonlinear optimization}

\section{Uncontrained nonlinear optimization}
\section{Constrained nonlinear optimization}
\subsection{Lagrange}
\subsection{Karush, Kuhn and Tucker theory}
\subsection{Contrained convex optimization}
\subsubsection{Frank and Wolfe's method}
\subsubsection{The cutting plane method}
\section{Lagrangian relaxation}
\section{Gradient methods}
\subsubsection{Fixed step gradient method}
\subsubsection{Optimal step gradient method}
\subsubsection{Conjugate gradient descent}
\subsection{Méthode de Newton}
\subsection{KKT Theory}
\subsection{Le Lagrangien}
\section{Optimisation continue stochastique}
\subsection{Méthode du gradient stochastique}
\subsection{Mirror descent}

\chapter{Stochastic optimization}

\section{Stochastic gradient descent}
\subsection{Setting and method}
\subsection{Theoretical bounds}
\subsection{Choosing the step}
\section{Particle SWARM optimization}

\chapter{Recent topics in optimization}

\section{Polynomial optimization}
% https://wangjie212.github.io/jiewang/research/lectures.pdf
\section{Quantum noncommutative polynomial optimization}
% ncpol3sdpa, examples, incluing the minecraft mobfarm
\section{Trust region algorithms}
% Yuan, Y. "A review of trust region algorithms for optimization" in ICIAM 99: Proceedings of the Fourth International Congress on Industrial & Applied Mathematics, Edinburgh, 2000 Oxford University Press, USA.
% Yuan, Y. "Recent Advances in Trust Region Algorithms", Math. Program., 2015
\subsection{The smoothed duality gap as a stopping criterion}
\section{Jacobi's algorithm (misc)}

\part{Discrete algorithmics}

\chapter{Complexity theory}
Complexity theory is an important domain of theoretical computer science, the most famous problem of which is probably the \og P = NP \fg problem. It asks whether or not NP problems are in P. Though we can, for many problems, proove that they \textit{are indeed} in a given problem complexity class, it is always very difficult to proove that a given problem is \textit{not} in some class.

We present the general theory, investigate the definitions along with their philosophical meanings, explore different complexity classes (co-NP, NPC, NPH, PH, ...) and their relationships. Next, we consider various algorithmic problems, study the relationships and reductions between them and some algorithms to solve them.

The references for this chapter are : \cite{gowers2023}, \cite{gowers2024}, \cite{hudry2024}, besides thoses cited herein.

\section{Theory}

\subsection{Turing machines and complexity}

Turing machines were originally introduced by A. Turing \cite{turing1936} \cite{turing1992}.

\subsubsection{The polynomial hierarchy}



Once $P$ and $NP$ have been defined,

\section{Examples, practice and problems}
\section{Examples, practice and problems}

% Nombreux exemples de problèmes (sack, 21 problèmes de Karp, ...)
% Reductions, SAT, HAM, Annales...

\part{Combinatorial optimization}

\chapter{Heuristics}\label{chap:heuristics}
\chapter{Meta-heuristics}\label{chap:meta-heuristics}

\part{Optimal transport}

\chapter{Optimal transport : general  results}

\chapter{Algorithmic approaches to optimal transport}

\section{Reduction to the simplex problem}
\section{The Eulerian point of view}
\section{Monge-Ampère's equation and applications}

\part{Problem solving}

\chapter{International Mathematics Olympiads (IMO) problems}

\section{IMO 2018 SL - Problem C1}
\section{IMO 2018 SL - Problem C2}
\section{IMO 2018 SL - Problem C3}
\section{IMO 2018 SL - Problem C4}
\section{IMO 2018 SL - Problem C5}
\section{IMO 2018 SL - Problem C6}

\section{IMO 2019 SL - Problem C1}
\section{IMO 2019 SL - Problem C2}
\section{IMO 2019 SL - Problem C3}
\section{IMO 2019 SL - Problem C4}
\section{IMO 2019 SL - Problem C5}
\section{IMO 2019 SL - Problem C6}

\chapter{Competitive programming problems}

\chapter{Graph problems}
% Un sujet ENS
\section{Combinatorics problems}

\bibliographystyle{plain}
\bibliography{bib/articles,bib/books,bib/misc}

\end{document}